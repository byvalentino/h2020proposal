%!TEX root = proposal.tex

\chapter{Ethics and security}
\label{cha:ethics}

\instructions{ \textit{This section is not covered by the page limit.} }

\section{Ethics}
\label{sec:ethics}

\instructions{
\textit{For more guidance, see the document ``How to complete your ethics self-assessment''.}

If you have entered any ethics issues in the ethical issue table in the
administrative proposal forms, you must:
\begin{itemize}
\item submit an ethics self-assessment, which:
\begin{itemize}
\item describes how the proposal meets the national legal and ethical
requirements of the country or countries where the tasks raising ethical issues
are to be carried out;
\item explains in detail how you intend to address the issues in the ethical
issues table, in particular as regard:
\begin{itemize}
\item research objectives (e.g., study of vulnerable populations, dual use,
etc.)
\item research methodology (e.g., clinical trials, involvement of children and
related consent procedures, protection of any data collected, etc.)
\item the potential impact of the research (e.g., dual use issues, environmental
damage, stigmatization of particular social groups, political or financial
retaliation, benefit-sharing,  malevolent use, etc.).
\end{itemize}
\end{itemize}
\item provide the documents that you need under national law (if you already
have them), e.g.:
\begin{itemize}
\item an ethics committee opinion;
\item the document notifying activities raising ethical issues or authorizing
such activities;
\end{itemize}
\end{itemize}
\textit{\indent If these documents are not in English, you must also submit an
English summary of them (containing, if available, the conclusions of the
committee or authority concerned).}

\textit{If you plan to request these documents specifically for the project you
are proposing, your request must contain an explicit reference to the project
title.} }

\lipsum[1]


\section[Security]{Security\instructions{\footnote{Article 37.1 of the Model Grant Agreement}}}
\label{sec:security}

LIPSUM will involve:
\begin{itemize}
\item activities or results raising security issues: \textbf{NO}
\item ``EU-classified information'' as background or results: \textbf{NO}
\end{itemize}
