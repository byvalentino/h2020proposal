%!TEX root = proposal.tex

\chapter{Implementation}
\label{cha:implementation}

\section{Work plan --- Work packages, deliverables}
\label{sec:work-plan}

\instructions{
Please provide the following:
\begin{itemize}
\item brief presentation of the overall structure of the work plan;
\item timing of the different Work Packages and their components (Gantt chart or
similar);
\item detailed work description, i.e.:
\begin{itemize}
\item a description of each Work Package (Table 3.1a);
\item a list of Work Packages (\autoref{tab:3.1b});
\item a list of major deliverables (\autoref{tab:3.1c});
\end{itemize}
\item graphical presentation of the components showing how they inter-relate
(Pert chart or similar).
\end{itemize}

\textit{Give full details. Base your account on the logical structure of the
project and the stages in which it is to be carried out. The number of work
packages should be proportionate to the scale and complexity of the project.}

\textit{ You should give enough detail in each Work Package to justify the
proposed resources to be allocated and also quantified information so that
progress can be monitored, including by the Commission}

\textit{ Resources assigned to Work Packages should be in line with their
objectives and deliverables. You are advised to include a distinct Work Package
on `management' (see section 3.2) and to give due visibility in the work plan to
`dissemination and exploitation' and `communication activities', either with
distinct tasks or distinct Work Packages.}

\textit{ You will be required to include an updated (or confirmed) `plan for the
dissemination and exploitation of results' in both the periodic and final
reports. (This does not apply to topics where a draft plan was not required.)
This should include a record of activities related to dissemination and
exploitation that have been undertaken and those still planned. A report of
completed and planned communication activities will also be required.}

\textit{If your project is taking part in the Pilot on Open Research Data, you
must include a  `data management plan' as a distinct deliverable within the
first 6 months of the project. A template for such a plan is given in the
guidelines on data management in the H2020 Online Manual.  This deliverable will
evolve during the lifetime of the project in order to present the status of the
project's reflections on data management.}

\textit{\textbf{Definitions:}}

\textit{\underline{``Work Package''} means a major sub-division of the proposed
project.}

\textit{\underline{``Deliverable''} means a distinct output of the project,
meaningful in terms of the project's overall objectives and constituted by a
report, a document, a technical diagram, a software etc.}}

\lipsum[1]

%!TEX root = proposal.tex

\makeatletter
\tikzset{%
  /pgfgantt/time slots/.code={%
    \tikzset{%
      /pgfgantt/time slots/.cd,
      #1,
      /pgfgantt/.cd,
    }%
  },
  /pgfgantt/time slots/.search also={/pgfgantt},
  /pgfgantt/time slots/.cd,
  width/.store in=\ts@width,
  width=\textwidth,
  slots/.store in=\totaltimeslots,
  slots=20,
  label width/.store in=\ts@labelwidth,
  label width=25mm,
  calc x unit/.code n args=3{%
    \pgfmathsetmacro\ts@xunit{(#1-#2-0.6667em-2*\pgflinewidth)/#3}%
    \tikzset{%
      /pgfgantt/x unit=\ts@xunit pt,
    }%
  },
  widest/.code={%
    \pgfmathsetmacro\ts@wdlabel{width("#1")}%
    \tikzset{/pgfgantt/time slots/label width=\ts@wdlabel pt}%
  },
  calc x unit aux width/.style={/pgfgantt/time slots/calc x unit={#1}{\ts@labelwidth}{\totaltimeslots}},
  calc x unit aux label width/.style={/pgfgantt/time slots/calc x unit={\ts@width}{#1}{\totaltimeslots}},
  calc x unit aux slots/.style={/pgfgantt/time slots/calc x unit={\ts@width}{\ts@labelwidth}{#1}},
  width/.forward to=/pgfgantt/time slots/calc x unit aux width,
  slots/.forward to=/pgfgantt/time slots/calc x unit aux slots,
  label width/.forward to=/pgfgantt/time slots/calc x unit aux label width,
}
\makeatother

\begin{figure}[!hb]
    \centering
    \begin{ganttchart}[
        time slots={slots=\arabic{@duration}, widest=TX.X},
        bar height=.8,
        bar inline label anchor=west,
        bar inline label node/.append style={right, color=white, font=\small},
        bar label font=\small,
        bar/.append style={draw=black, fill=black!30},
        % canvas/.append style={fill=none, draw=gray},
        group height=.8,
        group inline label anchor=west,
        group inline label node/.append style={right, color=white, font=\small},
        group label font=\bfseries\small,
        group left shift=0,
        group peaks height=.25,
        group peaks tip position=0,
        group right shift=0,
        group/.append style={draw=black, fill=black!70},
        milestone inline label node/.append style={font=\footnotesize},
        milestone top shift=.55,
        milestone/.append style={xscale=.55},
        title height=1,
        title label font=\footnotesize,
        % title/.append style={fill=none, draw=gray},
        vgrid,
        y unit chart=.55cm,
        y unit title=.55cm,
    ]{1}{\arabic{@duration}}
        \gantttitle{\acronym{} Project Month}{\arabic{@duration}} \\
        \gantttitlelist{1,...,\arabic{@duration}}{1} \\
        \ganttchartdata%
    \end{ganttchart}
\caption{\acronym{} Gantt chart.}
\label{fig:gantt}
\end{figure}


% Include work-packages as separate files
%!TEX root = proposal.tex

\begin{workpackage}{Project management}{wp:mgmt}

\wpstart{1}
\wpend{36}
\wptype{MGT}

\personmonths{COORD}{2}
\personmonths{TWO}{10}
\personmonths*{THREE}{21}

\makewptable{}

\begin{wpobjectives}
The high-level objectives of \autoref{wp:mgmt} are\dots
\end{wpobjectives}

\begin{wpdescription}

\wptask{COORD}{TWO}{1}{12}{Management task}{t:mgmt}{
\autoref{t:mgmt} is led by COORD\@. It is responsible for\dots
}

\wptask{THREE}{COORD}{12}{36}{Another task}{t:another}{
\autoref{t:another} is led by TWO and will work\dots
}

\end{wpdescription}


\begin{wpdeliverables}

% Type:
% Use one of the following codes:
% R:	Document, report (excluding the periodic and final reports)
% DEM:	Demonstrator, pilot, prototype, plan designs
% DEC:	Websites, patents filing,   press & media actions, videos, etc.
% OTHER: Software, technical diagram, etc.

% Dissemination level:
% Use one of the following codes:
% PU =	Public, fully open, e.g. web
% CO =	Confidential, restricted under conditions set out in Model Grant Agreement
% CI =	Classified, information as referred to in Commission Decision 2001/844/EC.

\wpdeliverable[1]{COORD}{R}{PU}{Project presentation}{}

This deliverable is a public description of the project.


\end{wpdeliverables}

\end{workpackage}

%!TEX root = proposal.tex

\begin{workpackage}{Technical work package}{wp:other}

\wpstart{6}
\wpend{36}
\wptype{RTD}

\personmonths{COORD}{2}
\personmonths*{TWO}{26}
\personmonths{THREE}{38}

\makewptable{}


\begin{wpobjectives}
\autoref{wp:other} will\dots
\end{wpobjectives}

\begin{wpdescription}

\wptask{TWO}{THREE}{6}{18}{Definition}{t:def}{
\autoref{t:def} will investigate\dots
}

\wptask{THREE}{TWO, COORD}{12}{36}{Design}{t:design}{
\autoref{t:design} will extend\dots
}

\end{wpdescription}


\begin{wpdeliverables}

% Type:
% Use one of the following codes:
% R:	Document, report (excluding the periodic and final reports)
% DEM:	Demonstrator, pilot, prototype, plan designs
% DEC:	Websites, patents filing,   press & media actions, videos, etc.
% OTHER: Software, technical diagram, etc.

% Dissemination level:
% Use one of the following codes:
% PU =	Public, fully open, e.g. web
% CO =	Confidential, restricted under conditions set out in Model Grant Agreement
% CI =	Classified, information as referred to in Commission Decision 2001/844/EC.

\wpdeliverable[6]{TWO}{R}{CO}{System architecture}{}

This restricted deliverable describes the architecture\dots


\end{wpdeliverables}

\end{workpackage}


\subsection{List of work packages}
\label{sec:wplist}
\makewplist%
% \maketasklist%

\subsection{List of deliverables}
\label{sec:deliverables}

\instructions{List of deliverables\footnote{If your action is taking part in the
Pilot on Open Research Data, you must include a data management plan as a
distinct deliverable within the first 6 months of the project.  This deliverable
will evolve during the lifetime of the project in order to present the status of
the project's reflections on data management. A template for such a plan is
available in the H2020 Online Manual on the Participant Portal.}}

\instructions{
\textbf{KEY}\\ \emph{Deliverable numbers in order of delivery dates. Please use
the numbering convention $<$WP number$>.<$number of deliverable within that
WP$>$.}

\textit{For example, deliverable 4.2 would be the second deliverable from work
package 4.}

\textbf{Type:}\\
\textit{Use one of the following codes:}\\
R\@: Document, report (excluding the periodic and final reports)\\
DEM\@: Demonstrator, pilot, prototype, plan designs\\
DEC\@: Websites, patents filing, press \& media actions, videos, etc.\\
OTHER\@: Software, technical diagram, etc.

\textbf{Dissemination level}:\\
\textit{Use one of the following codes:}\\
PU = Public, fully open, e.g., web\\
CO = Confidential, restricted under conditions set out in Model Grant Agreement\\
CI = Classified, information as referred to in Commission Decision 2001/844/EC.\\

\textbf{Delivery date}:\\
Measured in months from the project start date (month 1).
}

\makedeliverablelist%


\section{Management structure, milestones and procedures}
\label{sec:mgmt-struct}

\instructions{
\begin{itemize}
\item Describe the organisational structure and the decision-making (including a
list of milestones (\autoref{tab:3.2a}))
\item Explain why the organisational structure and decision-making mechanisms
are appropriate to the complexity and scale of the project.
\item Describe, where relevant, how effective innovation management will be
addressed in the management structure and work plan.
\\
\\
\textit{Innovation management is a process which requires an understanding of both
market and technical problems, with a goal of successfully implementing
appropriate creative ideas. A new or improved product, service or process is its
typical output. It also allows a consortium to respond to an external or
internal opportunity.}
\item Describe any critical risks, relating to project implementation, that the
stated project's objectives may not be achieved. Detail any risk mitigation
measures. Please provide a table with critical risks identified and mitigating
actions (\autoref{tab:3.2b})
\\
\\
\textit{\textbf{Definitions:}}
\\
\\
\textit{\underline{`Milestones'} means control points in the project that help to
chart progress. Milestones may correspond to the completion of a key
deliverable, allowing the next phase of the work to begin. They may also be
needed at intermediary points so that, if problems have arisen, corrective
measures can be taken. A milestone may be a critical decision point in the
project where, for example, the consortium must decide which of several
technologies to adopt for further development.}
\end{itemize}
}

\lipsum[1]



\subsection{Milestones}

\instructions{
\textbf{KEY}\\ \textbf{Estimated date}\\ \emph{Measured in months from the
project start date (month 1)}

\textbf{Means of verification}\\ \emph{Show how you will confirm that the
milestone has been attained. Refer to indicators if appropriate. For example: a
laboratory prototype that is `up and running'; software released and validated
by a user group; field survey complete and data quality validated.}}

% keep sorted by month

\milestone[1]{Successful project launch}{Meeting minutes}{WP\ref{wp:mgmt}}

\milestone[36]{Successful result}{Review report}{WP\ref{wp:mgmt}}

\makemilestoneslist{}


\subsection{Risk management}
\label{sec:risks}

\instructions{
\textbf{Definition critical risk:}\\ \emph{A critical risk is a plausible event
or issue that could have a high adverse impact on the ability of the project to
achieve its objectives.}

\textbf{Level of likelihood to occur: Low/medium/high}\\ \emph{The likelihood is
the estimated probability that the risk will materialize even after taking
account of the mitigating measures put in place.}}

\lipsum[1]

The foreseen project risks are described in \autoref{tab:3.2b}, below.

\criticalrisk{Partner leaving the consortium}{All}{Low}{Medium}{
The loss of a partner will be regulated by the Grant Agreement and the
Consortium Agreement.}

\makerisklist{}


\section{Consortium as a whole}
\label{sec:consortium}

\instructions{
\textit{The individual members of the consortium are described in a separate
section 4. There is no need to repeat that information here.}
\begin{itemize}
\item Describe the consortium. How will it match the project's objectives, and
bring together the necessary expertise? How do the members complement one
another (and cover the value chain, where appropriate),?
\item In what way does each of them contribute to the project? Show that each
has a valid role, and adequate resources in the project to fulfill that role.
\item If applicable, describe the industrial/commercial involvement in the
project to ensure exploitation of the results and explain why this is consistent
with and will help to achieve the specific measures which are proposed for
exploitation of the results of the project (see section 2.2).
\item \textbf{Other countries and international organizations:} If one or more
of the participants requesting EU funding is based in a country or is an
international organization that is not automatically eligible for such funding
(entities from Member States of the EU, from Associated Countries and from one
of the countries in the exhaustive list included in General Annex A of the work
programme are automatically eligible for EU funding), explain why the
participation of the entity in question is essential to carrying out the project
\end{itemize}
}

\lipsum[1]


\section{Resources to be committed}
\label{sec:resources}

\instructions{
\textit{Please make sure the information in this section matches the costs as
stated in the budget table in section 3 of the administrative proposal forms,
and the number of person/months, shown in the detailed Work Package
descriptions.}

Please provide the following:\\
\begin{itemize}
\item a table showing number of person/months required (\autoref{tab:3.4a})
\item a table showing ``other direct costs'' (\autoref{tab:3.4b}) for participants where those costs exceed \SI{15}{\percent} of the
personnel costs (according to the budget table in section 3 of the
administrative proposal forms)
\end{itemize}
}

\subsection{Summary of staff efforts}
\instructions{\autoref{tab:3.4a}: \emph{Please indicate the number of
person/months over the whole duration of the planned work, for each work
package, for each participant. Identify the work-package leader for each WP by
showing the relevant person-month figure in bold.}}

\makesummaryofefforttable%

\subsection[`Other direct cost' items]{`Other direct cost' items (travel, equipment, other goods and
services, large research infrastructure)}

\instructions{ Please provide a table of summary of costs for each participant,
if the sum of the costs for ``travel'', ``equipment'', and ``goods and
services'' exceeds \SI{15}{\percent} of the personnel costs for that participant (according
to the budget table in section 3 of the proposal administrative forms).}

\lipsum[1]

\costsTravel{TWO}{9999}{Travel costs for the project participation.}
\costsConsumables{TWO}{9999}{Pizza.}
\costsOther{TWO}{9999}{Open access payments.}

\costsTravel{THREE}{9999}{Travel costs for the project participation.}
\costsConsumables{THREE}{9999}{Pizza.}
\costsOther{THREE}{9999}{Open access payments.}

\makecoststable%

\instructions{
Please complete the table below for all participants that would like to declare
costs of large research infrastructure under Article 6.2 of the General Model
Agreement\footnote{Large research infrastructure means research infrastructure
of a total value of at least EUR 20 million, for a beneficiary. More information
and further guidance on the direct costing for the large research infrastructure
is available in the H2020 Online Manual on the Participant Portal.},
irrespective of the percentage of personnel costs. Please indicate (in the
justification) if the beneficiary's methodology for declaring the costs for
large research infrastructure has already been positively assessed by the
Commission.}

% \costslri{COORD}{400000}{Synchrotron}

% \makelritable%
