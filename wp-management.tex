%!TEX root = proposal.tex

\begin{workpackage}{Project management}{wp:mgmt}

\wpstart{1}
\wpend{36}
\wptype{MGT}

\personmonths{COORD}{2}
\personmonths{TWO}{10}
\personmonths*{THREE}{21}

\makewptable{}

\begin{wpobjectives}
The high-level objectives of \autoref{wp:mgmt} are\dots
\end{wpobjectives}

\begin{wpdescription}

\wptask{COORD}{TWO}{1}{12}{Management task}{t:mgmt}{
\autoref{t:mgmt} is led by COORD\@. It is responsible for\dots
}

\wptask{THREE}{COORD}{12}{36}{Another task}{t:another}{
\autoref{t:another} is led by TWO and will work\dots
}

\end{wpdescription}


\begin{wpdeliverables}

% Type:
% Use one of the following codes:
% R:	Document, report (excluding the periodic and final reports)
% DEM:	Demonstrator, pilot, prototype, plan designs
% DEC:	Websites, patents filing,   press & media actions, videos, etc.
% OTHER: Software, technical diagram, etc.

% Dissemination level:
% Use one of the following codes:
% PU =	Public, fully open, e.g. web
% CO =	Confidential, restricted under conditions set out in Model Grant Agreement
% CI =	Classified, information as referred to in Commission Decision 2001/844/EC.

\wpdeliverable[1]{COORD}{R}{PU}{Project presentation}{}

This deliverable is a public description of the project.


\end{wpdeliverables}

\end{workpackage}
